% $Log: abstract.tex,v $
% Revision 1.1  93/05/14  14:56:25  starflt
% Initial revision
% 
% Revision 1.1  90/05/04  10:41:01  lwvanels
% Initial revision
% 
%
%% The text of your abstract and nothing else (other than comments) goes here.
%% It will be single-spaced and the rest of the text that is supposed to go on
%% the abstract page will be generated by the abstractpage environment.  This
%% file should be \input (not \include 'd) from cover.tex.

Example: This is just a filler
I intend to further explore the grazer-prey interactions in the microbial loop, specifically which organisms are actively grazing throughout the year and which organisms they are exerting pressure on.  How these interactions change over time in a highly seasonal and variable environment remains fairly unknown. It is important to understand the dynamics of ciliate populations and their predator/prey relationships because of the major impacts they can have on primary production.  Microzooplankton have previously been assembled into functional groups because of methodological limitations (to increase sample size or because they cannot be identified), which is not necessarily appropriate. Size categories may not reflect similar prey ingestion rates or growth efficiencies, and taxonomy may not reflect similar prey preferences or growth parameters (some ciliates can feed on diatoms much larger than themselves, while others in the same size range ingest tiny cells). Many laboratory investigations have been conducted on species that may not be environmentally active or abundant and while they provide some information about grazing rates, we do not know whether these organisms are abundant in natural samples. To study interactions on these levels, I will combine new observational methods exploring community structure down to individuals with molecular techniques supplementing the scale of organism to organism interactions.  
